%!TEX program=xelatex
\documentclass[UTF-8, a4paper, 12pt]{ctexart}
\setlength{\parindent}{0pt}
\usepackage{setspace}
\renewcommand{\baselinestretch}{1.5}
\usepackage{tikz}
\usepackage{pgfplots}
\usepackage{textcomp}
\usepackage[left=2.50cm, right=2.50cm, top=2.50cm, bottom=2.50cm]{geometry}
\usepackage{fancyhdr}
\usepackage{enumerate}
\pagestyle{fancy}
\rhead{\center{\small{华南理工大学大学城校区物理实验报告}}}

\begin{document}
    \begin{center}
        \zihao{-2}\heiti{模拟示波器的调节与使用 预习报告}

        \zihao{-4}\songti{2020级 \quad 计算机科学与技术(全英创新班)\quad 王樾}
    \end{center}
    \zihao{4}\heiti{引言:}\zihao{-4}\songti{示波器是一种用途广泛的电子测量仪器。示波器能通过直接或间接的测量手段,测量出电压、电流、阻抗、压力、磁场等物理量关于时间的变化状况。通过示波器上的图像,我们不仅能够直观地观察被测信号的变化规律,还能测量出其他的各种表征参数。学习了模拟示波器的原理和使用方法,就能为各类示波器的使用打下基础。}

    \textbf{\zihao{4} \heiti{一、实验目的}}

    \zihao{-4}\songti

    \begin{enumerate}[(1)]
        \item 了解示波器的基本结构和工作原理,掌握示波器的使用方法。
        \item 学会用示波器观察电信号波形。
        \item 学会用示波器测量交流电信号电压峰-峰值($V_{p-p}$)和频率。
    \end{enumerate}

    \textbf{\zihao{4} \heiti{二、实验仪器}}

    \zihao{-4}\songti

    示波器由示波管、衰减器和放大器、扫描信号发生器、同步触发系统和电源供给系统五个基本部分组成。示波关主要由电子枪、偏转系统和荧光屏三部分组成,是一个全密封的真空玻璃管壳。衰减器和放大器均有x轴和y轴两种类别,适用于处理输入信号电压过大或过小的情况。扫描信号发生器在水平偏转板加上“锯齿波”状的扫描信号。同步触发系统保证锯齿波的周期与待测信号周期一致,使示波器上的波形保持稳定。

    \textbf{\zihao{4} \heiti{三、实验原理}}

    \zihao{-4}\songti

    将待测信号电压加在控制垂直偏转的电路中,同时把水平扫描信号电压加在控制水平的电路。电子枪发射的电子束通过这两个电路时,会因电信号的变化而改变运动轨迹。最后电子轰击荧光屏上的荧光粉发光,就得到了待测电信号的波形图像。

    \textbf{\zihao{4} \heiti{四、内容步骤}}

    \zihao{-4}\songti

    \begin{enumerate}[(1)]
        \item 学习通用示波器的按钮作用以及使用方法。
        \item 依次将8个波形信号输入示波器,记录档位示值、波形周期宽度及峰-峰高度。
        \item 在示波器校准后,用定标法测量交流电桥信号源的频率$f$及电压峰-峰值$V_{p-p}$。
        \item 利用李萨如图形,绘下某一时刻的图形,记录低频信号发生器显示的输出频率$f_x$。
    \end{enumerate}

\end{document}