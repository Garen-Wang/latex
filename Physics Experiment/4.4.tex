%!TEX program=xelatex
\documentclass[UTF-8, a4paper, 12pt]{ctexart}
\setlength{\parindent}{0pt}
\usepackage{setspace}
\renewcommand{\baselinestretch}{1.5}
\usepackage{tikz}
\usepackage{pgfplots}
\usepackage{textcomp}
\usepackage[left=2.50cm, right=2.50cm, top=2.50cm, bottom=2.50cm]{geometry}
\usepackage{fancyhdr}
\usepackage{enumerate}
\pagestyle{fancy}
\rhead{\center{\small{华南理工大学大学城校区物理实验报告}}}

\begin{document}
    \begin{center}
        \zihao{-2}\heiti{PN结正向电压温度特性 预习报告}
        

        \zihao{-4}\songti{2020级 \quad 计算机科学与技术(全英创新班)\quad 王樾}
    \end{center}
    \zihao{4}\heiti{引言:}\zihao{-4}
    \songti
    常见的温度传感器有热电偶、热敏电阻、测温电阻等。热电偶适用温度范围宽,但灵敏度低、线性差,热敏电阻灵敏度高,体积小,但线性差,测温电阻精度高、线性好,但灵敏度低且价格昂贵。而PN结温度传感器具有灵敏度好、热响应快、线性好、体积小、轻便等特点,在许多方面优于其他传感器,所以应用广泛。

    \textbf{\zihao{4} \heiti{一、实验目的}}

    \zihao{-4}\songti

    \begin{enumerate}[(1)]
        \item 了解PN结正向电压随温度变化的基本规律。
        \item 在电源恒流条件下,测绘PN结正向电压随温度变化的关系曲线,并确定PN结的测温灵敏度和PN结材料的禁带宽度。
    \end{enumerate}

    \textbf{\zihao{4} \heiti{二、实验仪器}}

    \zihao{-4}\songti

    PN结正向特性综合实验仪、DH-SJ5温度传感器实验装置。

    \textbf{\zihao{4} \heiti{三、实验原理}}

    \zihao{-4}\songti

    理想PN结的正向电流$I_F$和正向电压$V_F$存在如下近似关系:

    $$I_F = I_n e^{\frac{q V_F}{kT}}$$

    可以证明$$I_n = CT^\gamma e^{\frac{qV_{g_0}}{kT}}$$

    其中,$C$是PN结的一个常数,$k$为玻尔兹曼常数,$\gamma$在一定温度范围内也是常数。

    联立得

    $$V_F = V_{g_0} - (\frac k q \ln \frac{C}{I_F})T - \frac{kT}{q}\ln T^\gamma = V_1 + V_{nr}$$

    其中$$V_1 = V_{g_0} - (\frac k q \ln \frac{C}{I_F})T, V_{nr} = - \frac{kT}{q}\ln T^\gamma$$.

    当恒流供电(即$I_F$不变)条件下,$V_{nr}$不变,此时PN结的正向电压$V_F$变化取决于线性项$V_1$,$V_F$非常近似于随温度升高而线性下降。

    记斜率绝对值$\frac k q \ln \frac{C}{I_F}$为$S$,定义其为PN结温度灵敏度,记$t$为摄氏温度,则有$$\Delta V = S t$$

    PN结材料的禁带宽度$E_{g_0}$定义为电子电量$q$与$0K$时PN结材料的导带底和价带顶的电势差$V_{g_0}$的乘积,即$E_{g_0} = q V_{g_0}$。由上式得

    $$V_{g_0} = V_F + (\frac k q \ln \frac{C}{I_F})T = V_F + S T$$

    代入$0^\circ C$时,$T = 273.2K, V_F = V_{F_0}$,有

    $$E_{g_0} = qV_{g_0} = q(V_{F_0} + 273.2S)$$

    通过该式即可求出PN结的禁带宽度。

    \textbf{\zihao{4} \heiti{四、内容步骤}}

    \zihao{-4}\songti

    \begin{enumerate}[(1)]
        \item 首先将DH-SJ型温度传感器实验装置的“加热电流”与“风扇电流”置于关状态,接上电源线。插好Pt100温度传感器和PN结温度传感器。打开开关,显示出室温$t_r$,记录起始温度$t_r$。
        \item 记录同一温度下正向电压随正向电流的变化关系。将电流量程置于"$\times 1$"档,调节电流调节旋钮,观察$V_F$值读数的变化(如果电流表显示值大于1000则可改用大一档量程)。记录一系列电压、电流值于表。
        \item 记录在同一恒定电流条件下PN正向电压与温度的变化关系。首先选择合适的正向电流$I_F$(如$I_F = 60\mu A$)并保持不变。
        \item 将温度传感器实验装置的“加热电流”开关打开,根据目标温度,选择合适的加热电流。条件允许时电流可以取得小一点,如$0.3~0.6A$之间。
        \item 在加热过程中,记录对应的$V_F$和$T$于表。
    \end{enumerate}

\end{document}