%!TEX program=xelatex
\documentclass[UTF-8, a4paper, 12pt]{ctexart}
\setlength{\parindent}{0pt}
\usepackage{setspace}
\renewcommand{\baselinestretch}{1.5}
\usepackage{tikz}
\usepackage{pgfplots}
\usepackage{textcomp}
\usepackage[left=2.50cm, right=2.50cm, top=2.50cm, bottom=2.50cm]{geometry}
\usepackage{fancyhdr}
\usepackage{enumerate}
\pagestyle{fancy}
\rhead{\center{\small{华南理工大学大学城校区物理实验报告}}}

\begin{document}
    \begin{center}
        \zihao{-2}\heiti{液体动力黏度的测量 预习报告}

        \zihao{-4}\songti{2020级 \quad 计算机科学与技术(全英创新班)\quad 王樾}
    \end{center}
    \zihao{4}\heiti{引言:}\zihao{-4}
    \songti
    在稳定流动的液体中,液体质元之间存在的内摩擦力产生了对流动的阻抗,这种性质称为黏滞性。我们可以通过黏滞系数表示液体、气体等流体的黏性。在实际工程和工业生产中,经常需要检测流体的黏度以保证最佳的运行环境和产品质量,从而提高生产效益。本实验探索用奥氏黏度计来测量酒精的动力黏度。

    \textbf{\zihao{4} \heiti{一、实验目的}}

    \zihao{-4}\songti

    \begin{enumerate}[(1)]
        \item 掌握用奥氏黏度计测定液体动力黏度的方法。
        \item 掌握秒表、量杯、温度计的基本操作。
        \item 进一步理解液体黏滞性的意义。
    \end{enumerate}

    \textbf{\zihao{4} \heiti{二、实验仪器}}

    \zihao{-4}\songti

    奥氏黏度计、温度计、比重计、秒表、酒精、蒸馏水、移液管、吸球、玻璃缸、支架、胶管。

    \textbf{\zihao{4} \heiti{三、实验原理}}

    \zihao{-4}\songti

    根据泊肃叶定律,液体流经毛细管时,体积流量$Q$与管的两端压强$\Delta p$,管的半径$r$,长度$L$以及黏滞系数$\eta$有如下关系:

    $$Q = \frac{V}{t} = \frac{\pi r^4 \Delta p}{8\eta L}$$

    根据上式可得$$\frac{\eta}{t\Delta p} = \frac{\pi r^4}{8VL}$$
    
    设已知蒸馏水黏度为$\eta_1$,待测酒精黏度为$\eta_2$,在半径与长度相同的毛细管中,使两种液体上升至同一高度$A$处,分别测出两种液体从$A$降到同一高度$B$所需时间$t_1$和$t_2$。通过控制变量,由上式可得:

    $$\frac{\eta_1}{t_1\Delta p_1} = \frac{\eta_2}{t_2\Delta p_2} \iff \frac{\eta_1}{\eta_2} = \frac{t_1\Delta p_1}{t_2 \Delta p_2}$$

    由于液体压强差可由$\Delta p=\rho g \Delta h$算出,则$$\frac{\Delta p_1}{\Delta p_2} = \frac{\rho_1}{\rho_2}$$

    则待测的酒精黏度可表示为$$\eta_2 = \frac{t_2\Delta p_2}{t_1\Delta p_1}\eta_1 = \frac{\rho_2 t_2}{\rho_1 t_1}\eta_1$$

    用秒表测量$t_1$和$t_2$,用比重计测出$\rho_1$和$\rho_2$,那么酒精的黏度系数即可用已知的$\eta_1$表示。

    \textbf{\zihao{4} \heiti{四、内容步骤}}

    \zihao{-4}\songti

    \begin{enumerate}[(1)]
        \item 用蒸馏水清理黏度计内部,然后将黏度计装好浸在盛水的玻璃缸中,让水面超过痕线$A$。
        \item 用移液管将一定量的蒸馏水注入右边的泡中。
        \item 用吸球将蒸馏水吸入左边的泡中并使其液面略高于痕线$A$,然后让液体自然从毛细管流下,当液面降至A时开始计时,降至B时停止计时,记下时间$t_1$。重复测量三次以上。将数据记录于表中。
        \item 将蒸馏水替换为相同体积的待测酒精,重复第(2)步和第(3)步,测得时间$t_2$,同样需重复测量三次以上。将数据记录于表中。
        \item 每次测$t_1$和$t_2$时,记下初次和末次时玻璃缸中的水温,用密度计分别测定蒸馏水和酒精的密度$\rho_1$和$\rho_2$。
        \item 根据水温查得蒸馏水的动力黏度$\eta_1$,进行数据处理,算出酒精的粘滞系数$\eta_2$。
    \end{enumerate}

\end{document}