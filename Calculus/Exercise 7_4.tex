\documentclass{article}
\usepackage{amsmath}
\usepackage{amsthm}
\usepackage{amssymb}
\usepackage{enumerate}
\usepackage{graphicx}
%done
\begin{document}
    \title{Exercise 7\_4}
    \author{Wang Yue from CS Elite Class}
    \date{\today}
    \maketitle

    \section*{Exercise 7.4}

    \subsection*{11. $\int_0^1 \frac{2}{2x^2 + 3x + 1} dx$}

    $$\begin{aligned}
        \int_0^1 \frac{2}{2x^2 + 3x + 1} dx &= \int_0^1 \frac{2}{(2x + 1)(x + 1)} dx \\
        &= \int_0^1 (\frac{A}{2x + 1} + \frac{B}{x + 1}) dx
    \end{aligned}$$

    $$A(x + 1) + B(2x + 1) = (A + 2B) x + (A + B) = 2$$

    Solving $\left\{ \begin{array}{ll}
        A + 2B = 0 \\
        A + B = 2
    \end{array} \right.$, we get $\left\{ \begin{array}{ll}
        A = 4 \\
        B = -2
    \end{array} \right.$

    $$\begin{aligned}
        \therefore \int_0^1 \frac{2}{2x^2 + 3x + 1} dx &= \int_0^1 (\frac{4}{2x + 1} - \frac{2}{x + 1}) dx \\
        &= (2\ln (2x + 1) - 2 \ln(x + 1)) \biggl|_0^1 \\
        &= 2(\ln 3 - \ln 2 - \ln 1 + \ln 1) \\
        &= 2\ln \frac 3 2
    \end{aligned}$$

    \subsection*{19. $\int \frac{x^2 + 1}{(x - 3)(x - 2)^2} dx$}

    $$\begin{aligned}
        \int \frac{x^2 + 1}{(x - 3)(x - 2)^2} dx &= \int (\frac{A}{x - 3} + \frac{B}{x - 2} + \frac{C}{(x - 2)^2}) dx \\
    \end{aligned}$$

    $$A(x - 2)^2 + B(x - 3)(x - 2) + C(x - 3) = x^2 + 1$$

    Let $x = 2$, then $-C = 5 \iff C = -5$

    Let $x = 3$, then $A = 10$

    Let $x = 4$, then $40 + 2B - 5 = 17 \iff B = -9$

    $$\begin{aligned}
        \therefore \int \frac{x^2 + 1}{(x - 3)(x - 2)^2} dx &= \int (\frac{10}{x - 3} - \frac{9}{x - 2} - \frac{5}{(x - 2)^2}) dx \\
        &= 10 \ln |x - 3| - 9\ln |x - 2| - \int \frac{5}{(x - 2)^2} d(x - 2) \\
        &= 10\ln |x - 3| - 9\ln |x - 2| + \frac{5}{x - 2} + K
    \end{aligned}$$

    \subsection*{24. $\int \frac{x^2 - x + 6}{x^3 + 3x} dx$}

    $$\begin{aligned}
        \int \frac{x^2 - x + 6}{x^3 + 3x} dx &= \int (\frac{A}{x} + \frac{Bx + C}{x^2 + 3}) dx \\
    \end{aligned}$$

    $$A(x^2 + 3) + (Bx + C)x = (A + B)x^2 + Cx + 3A = x^2 - x + 6$$

    $\therefore \left\{ \begin{array}{ll}
        A = 2 \\
        B = -1 \\
        C = -1 \\
    \end{array} \right.$

    $$\begin{aligned}
        \therefore \int \frac{x^2 - x + 6}{x^2 + 3x} dx &= \int (\frac{2}{x} + \frac{-x - 1}{x^2 + 3}) dx \\
        &= 2\ln|x| - \int \frac{x}{x^2 + 3} dx - \int \frac{1}{x^2 + 3} dx \\
        &= 2\ln|x| - \frac 1 2 \int \frac{dx^2}{x^2 + 3} - \int \frac{1}{x^2 + 3} dx \\
        &= 2\ln|x| - \frac 1 2 \ln |x^2 + 3| - \int \frac{1}{x^2 + 3} dx
    \end{aligned}$$

    Let $x = \sqrt 3 u, x^2 = 3u^2$, then 

    $$\begin{aligned}
        \int \frac{1}{x^2 + 3} dx &= \frac{\sqrt 3}{3} \int \frac{1}{u^2 + 1} du \\
        &= \frac{1}{\sqrt 3}\arctan u + K \\
        &= \frac{1}{\sqrt 3}\arctan \frac{x}{\sqrt 3} + K
    \end{aligned}$$

    $$\therefore \int \frac{x^2 - x + 6}{x^2 + 3x} dx = 2\ln|x| - \frac{\ln|x^2 + 3|}{2} - \frac{1}{\sqrt 3} \arctan \frac{x}{\sqrt 3} + K$$

    \subsection*{26. $\int \frac{x^2 + x + 1}{(x^2 + 1)^2} dx$}

    $$\begin{aligned}
        \int \frac{x^2 + x + 1}{(x^2 + 1)^2} dx &= \int (\frac{1}{x^2 + 1} + \frac{1}{(x^2 + 1)^2}) dx \\
        &= \int \frac{1}{x^2 + 1} dx + \int \frac{1}{(x^2 + 1)^2} dx \\
        &= \arctan x + \frac 1 2 \int \frac{d(x^2 + 1)}{(x^2 + 1)^2} \\
        &= \arctan x - \frac{1} {2(x^2 + 1)} + C
    \end{aligned}$$

    \subsection*{28. $\int \frac{x^2 - 2x - 1}{(x - 1)^2(x^2 + 1)}dx$}

    $$\begin{aligned}
        \int \frac{x^2 - 2x - 1}{(x - 1)^2(x^2 + 1)}dx &= \int (\frac{A}{x - 1} + \frac{B}{(x - 1)^2} + \frac{Cx + D}{x^2 + 1})dx
    \end{aligned}$$

    $$A(x - 1)(x^2 + 1) + B(x^2 + 1) + (Cx + D)(x - 1)^2 = x^2 - 2x - 1$$

    % $$(x^3 + x - x^2 - 1)A + B(x^2 + 1) + (Cx^3 - 2Cx^2 + Cx) + (Dx^2 - 2Dx + D)$$

    $$(A + C)x^3 + (-A + B - 2C + D) x^2 + (A + B + C - 2D) x + (-A + B + D) = x^2 - 2x - 1$$

    Solving $\left\{ \begin{array}{ll}
        A + C = 0 \\
        -A + B - 2C + D = 1 \\
        A + B + C - 2D = -2 \\
        -A + B + D = -1
    \end{array} \right.$, we have $\left\{ \begin{array}{ll}
        A = 1 \\
        B = -\frac 2 3 \\
        C = -1 \\
        D = \frac 2 3
    \end{array} \right.$

    $$\begin{aligned}
        \therefore \int \frac{x^2 - 2x - 1}{(x - 1)^2(x^2 + 1)}dx &= \int(\frac{1}{x - 1} - \frac{\frac 2 3}{(x - 1)^2} + \frac{-x + \frac 2 3}{x^2 + 1}) dx \\
        &= \int\frac{1}{x - 1} dx - \int \frac{\frac 2 3}{(x - 1)^2}dx + \int \frac{-x + \frac 2 3}{x^2 + 1} dx \\
        &= \ln|x - 1| - \frac 2 3 \int \frac{d(x - 1)}{(x - 1)^2} - \int \frac{x}{x^2 + 1} dx + \frac 2 3 \int \frac{1}{x^2 + 1} dx \\
        &= \ln|x - 1| + \frac{2}{3(x - 1)} - \frac 1 2 \int \frac{dx^2}{x^2 + 1} + \frac 2 3 \arctan x \\
        &= \ln|x - 1| + \frac{2}{3(x - 1)} - \frac{\ln|x^2 + 1|}{2} + \frac 2 3 \arctan x + K
    \end{aligned}$$

    \subsection*{49. $\int \frac{\sec^2 t}{\tan^2 t + 3\tan t + 2} dt$}

    $$\begin{aligned}
        \int \frac{\sec^2 t}{\tan^2 t + 3\tan t + 2} dt &= \int \frac{d \tan t}{(\tan t + 1)(\tan t + 2)} \\
        &= \int (\frac{1}{\tan t + 1} - \frac{1}{\tan t + 2}) d \tan t \\
        &= \ln|\tan t + 1| - \ln |\tan t + 2| + C
    \end{aligned}$$

    \subsection*{50. $\int \frac{e^x}{(e^x - 2)(e^{2x} + 1)} dx$}

    Let $u = e^x, x = \ln u$, then

    $$\begin{aligned}
        \int \frac{e^x}{(e^x - 2)(e^{2x} + 1)} dx &= \int \frac{1}{(u - 2)(u^2 + 1)}du \\
        &= \int (\frac{A}{u - 2} + \frac{Bu + C}{u^2 + 1}) du \\
    \end{aligned}$$

    $$A(u^2 + 1) + (Bu + C)(u - 2) = (A + B)u^2 + (C - 2B)u + A - 2C = 1$$

    Solving $\left\{ \begin{array}{ll}
        A + B = 0 \\
        C - 2B = 0 \\
        A - 2C = 1
    \end{array} \right.$, we can get $\left\{ \begin{array}{ll}
        A = \frac 1 5 \\
        B = -\frac 1 5 \\
        C = -\frac 2 5
    \end{array} \right.$

    $$\begin{aligned}
        \therefore \int \frac{e^x}{(e^x - 2)(e^{2x} + 1)} dx &= \frac 1 5 \int(\frac{1}{u - 2} - \frac{u + 2 }{u^2 + 1}) du \\
        &= \frac 1 5(\ln|u - 2| - \int \frac{u}{u^2 + 1} du - \int \frac{2}{u^2 + 1} du) \\
        &= \frac 1 5(\ln|u - 2| - \frac 1 2 \int \frac{du^2}{u^2 + 1} - 2\arctan u) \\
        &= \frac 1 5\ln|u - 2| - \frac{1}{10} \ln|u^2 + 1| - \frac 2 5 \arctan u + K \\
        &= \frac 1 5 \ln|e^x - 2| - \frac{1}{10}\ln|e^{2x} + 1| - \frac 2 5 \arctan e^x + K
    \end{aligned}$$

    \subsection*{61. $\int \frac{1}{3\sin x - 4\cos x} dx$}

    Let $t = \tan \frac x 2, x = 2\arctan t$, then $\sin x = \frac{2t}{1 + t^2}, \cos x = \frac{1 - t^2}{1 + t^2}$, then

    $$\begin{aligned}
        \int \frac{1}{3\sin x - 4\cos x} dx &= \int \frac{1}{\frac{6t - 4(1 - t^2)}{1 + t^2}}\frac{2}{1 + t^2} dt  \\
        &= \int \frac{1}{2t^2 + 3t - 2}dt \\
        &= \int \frac{1}{(2t - 1)(t + 2)}dt \\
        &= \int (\frac{A}{2t - 1} + \frac{B}{t + 2}) dt
    \end{aligned}$$

    Solving $At + 2A + 2Bt - B = (A + 2B)t + 2A - B = 1$, we get $$A = \frac 2 5, B = -\frac 1 5$$

    $$\begin{aligned}
        \therefore \int \frac{1}{3\sin x - 4\cos x} dx &= \frac 1 5 \int(\frac{2}{2t - 1} - \frac{1}{t + 2}) dt \\
        &= \frac 1 5(\ln|2t - 1| - \ln|t + 2|) + C \\
        &= \frac 1 5 \ln|\frac{2t - 1}{t + 2}| + C
    \end{aligned}$$
    
    \subsection*{62. $\int_{\pi/3}^{\pi/2} \frac{1}{1 + \sin x - \cos x} dx$}

    Let $t = \tan \frac x 2, x = 2\arctan t$, then $\sin x = \frac{2t}{1 + t^2}, \cos x = \frac{1 - t^2}{1 + t^2}$, then

    $$\begin{aligned}
        \int_{\pi/3}^{\pi/2} \frac{1}{1 + \sin x - \cos x} dx &= \int_{\frac{1}{\sqrt 3}}^{1} \frac{1}{1 + \frac{2t - 1 + t^2}{1 + t^2}} \frac{2}{1 + t^2} dt \\
        &= \int_{\frac{\sqrt 3}{3}}^1 \frac{2}{2t^2 + 2t}dt \\
        &= \int_{\frac{\sqrt 3}{3}}^1 \frac{1}{t} - \frac{1}{t + 1} dt \\
        &= (\ln|\frac{t}{t + 1}|)\biggl|_{\frac{\sqrt 3}{3}}^1 \\
        &= \ln\frac 1 2 - \ln \frac{1}{1 + \sqrt 3} \\
        &= \ln \frac{1 + \sqrt 3}{2}
    \end{aligned}$$

    \subsection*{72. If $f$ is a quadratic function such that $f(0) = 1$ and $$\int \frac{f(x)}{x^2(x + 1)^3}dx $$ is a rational function, find the value of $f'(0)$.}

    Let $f(x) = ax^2 + bx + 1$, then $f'(x) = 2ax + b, f'(0) = b$, so we only need to evaluate the value of $b$.
    
    $$\int \frac{ax^2 + bx + 1}{x^2(x + 1)^3} dx = \int (\frac{A}{x} + \frac{B}{x^2} + \frac{C}{x + 1} + \frac{D}{(x + 1)^2} + \frac{E}{(x + 1)^3}) dx $$

    To make $\int \frac{ax^2 + bx + 1}{x^2(x + 1)^3} dx$ is a rational function, $\frac A x$ and $\frac{C}{x + 1}$ cannot appear in the integrand, which means $$A = C = 0$$

    $$\therefore \int \frac{ax^2 + bx + 1}{x^2(x + 1)^3} dx = \int (\frac{B}{x^2} + \frac{D}{(x + 1)^2} + \frac{E}{(x + 1)^3}) dx $$

    $$\begin{aligned}
        B(x + 1)^3 + D(x + 1)x^2 + Ex^2 &= (B + D)x^3 + (3B + D + E)x^2 + (3B)x + B \\ 
        &= ax^2 + bx + 1
    \end{aligned}$$

    Solving $\left\{ \begin{array}{ll}
        B + D = 0 \\
        3B + D + E = a \\
        3B = b \\
        B = 1
    \end{array} \right.$, we can know $b = 3$.

    $\therefore f'(0) = b = 3$

\end{document}