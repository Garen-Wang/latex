\documentclass{article}
\usepackage{amsmath}
\usepackage{amsthm}
\usepackage{amssymb}
\usepackage{enumerate}
\usepackage{graphicx}

% done
\begin{document}
    \title{MA 5\_4 Exercise}
    \author{Wang Yue from CS Elite Class}
    \date{\today}
    \maketitle
    
    \section*{Exercise 5.5}

    \subsection*{30. $\int \frac{\tan ^{-1} x }{1 + x^2} dx$}

    $$\begin{aligned}
        \int \frac{\arctan x}{1 + x^2}dx &= \int \arctan x d\arctan x \\
        &= \frac{\arctan^2 x}{2} + C
    \end{aligned}$$

    \subsection*{31. $\int e^{\tan x} \sec^2 x dx$}

    $$\begin{aligned}
        \int e^{\tan x} \sec^2 x dx &= \int e^{\tan x} d\tan x \\
        &= e^{\tan x} + C
    \end{aligned}$$

    \subsection*{34. $\int \frac{\cos(\frac \pi x)}{x^2} dx$}

    $$\begin{aligned}
        \int \frac{\cos(\frac \pi x)}{x^2} dx &= -\frac{1}{\pi} \int \cos (\frac \pi x) d\frac \pi x \\
        &= -\frac 1 \pi (-\sin \frac \pi x) + C \\
        &= \frac 1 \pi \sin \frac \pi x + C
    \end{aligned}$$

    \subsection*{46. $\int x^2\sqrt{2 + x} dx$}

    Let $t = x + 2$, then 
    $$\begin{aligned}
        \int x^2\sqrt{2 + x}dx &= \int(t - 2)^2\sqrt t dt \\
        &= \int (t^{\frac 5 2} - 4t^{\frac 3 2} + 4t^{\frac 1 2})dt \\
        &= \frac 2 7 x^{\frac 7 2} - \frac 8 5 x^{\frac 5 2} + \frac 8 3 x^{\frac 3 2} + C \\
    \end{aligned}$$


    \subsection*{66. $\int_{-\frac \pi 3}^{\frac \pi 3}x^4\sin x dx$}

    $\because x \in [-\frac \pi 3, \frac \pi 3], y = x^4\sin x$ is an odd function

    $\therefore \int_{-\frac \pi 3}^{\frac \pi 3}x^4\sin x dx = 0$

    \subsection*{69. $\int_e^{e^4} \frac{dx}{x\sqrt{\ln x}}$}

    $$\begin{aligned}
        \int_e^{e^4}\frac{dx}{x\ln \sqrt x} &= \int_e^{e^4} \frac{2d\ln \sqrt x}{\ln \sqrt x}  \\
        &= (2 \ln \ln \sqrt x) \biggl|_e^{e^4} \\
        &= 2(\ln 4 - \ln 1) \\
        &= 4\ln 2
    \end{aligned}$$

    \subsection*{85. If $f$ is continuous and $\int_0^4f(x)dx = 10$, find $\int_0^2f(2x)dx$}

    Let $u = 2x$, then $$\int_0^2f(2x)dx = \frac 1 2 \int_0^4f(u)du = \frac{10}{2} = 5$$

    \subsection*{86. If $f$ is continuous and $\int_0^9f(x)dx = 4$, find $\int_0^3xf(x^2)dx$}

    Let $u = x^2$, then $$\int_0^3xf(x^2)dx = 2 \int_0^9f(u)du = 2 \times 4 = 8$$

    \section*{Exercise 7.2}

    \subsection*{1. $\int \sin^3 x \cos^2 x dx$}

    $$\begin{aligned}
        \int \sin^3 x \cos^2 x dx &= \int \sin^3 x (1 - \sin^2 x) dx \\
        &= \int \sin^3 x - \sin^5 x dx \\
        &= -\frac 1 3 \cos^3 x + \frac 1 5 \cos^5 x + C
    \end{aligned}$$

    \subsection*{6. $\int \frac{\sin^3(\sqrt x)}{\sqrt x} dx$}

    $$\begin{aligned}
        \int \frac{\sin^3(\sqrt x)}{\sqrt x} dx &= \int 2\sin^3(\sqrt x)\frac{1}{2\sqrt x}dx  \\
        &= \int 2\sin^3 (\sqrt x)d\sqrt x \\
        &= -\frac 2 3\cos^3 (\sqrt x) + C
    \end{aligned}$$

    % \subsection*{7. $\int_0^{\frac \pi 2}\cos^2 \theta d\theta$}

    % $$\begin{aligned}
    %     \int_0^{\frac \pi 2}\cos^2 \theta d\theta &= \int_0^{\frac \pi 2}\frac{1 + \cos 2\theta}{2}d\theta \\
    %     &= (\frac \theta 2 + \frac{\sin 2\theta}{4})\biggl|_0^{\frac \pi 2}\\
    %     &= \frac \pi 4
    % \end{aligned}$$

    \subsection*{16. $\int x\sin^3 x dx$} 

    $$\begin{aligned}
         \int x\sin^3 x dx &= \int x \sin x (1 - \cos ^2 x) dx \\
         &= \int (x\sin x - x \cos^2 x \sin x)dx \\
         &= \int x\sin x dx - \int x \cos^2 x \sin x dx \\
         &= -\int xd\cos x + \frac 1 3\int x d\cos ^3 x \\
         &= -x\cos x + \int \cos x dx + \frac 1 3(x \cos ^3 x - \int cos^3 x dx) \\
         &= -x\cos x + \sin x + \frac 1 3 x \cos ^3 x - \frac 1 3 \int (\cos x - \cos x \sin^2 x) dx  \\
         &= -x\cos x + \sin x + \frac 1 3 x \cos ^3 x - \frac 1 3 (\sin x - \frac 1 3 \sin ^3 x) + C \\
         &= \frac 1 3 x \cos ^3 x + \frac 2 3 \sin x - x \cos x + \frac 1 9 \sin^3 x + C
    \end{aligned}$$

    \subsection*{21. $\int \tan x \sec ^3 x dx$}

    $$\begin{aligned}
        \int \tan x \sec^3 x dx &= \int \sec^2 x (\tan x \sec x dx) \\
        &= \int \sec^2 x d\sec x \\
        &= 2\sec^2 x + C
    \end{aligned}$$

    \subsection*{23. $\int \tan^2 x dx$}

    $$\begin{aligned}
        \int \tan^2 x dx &= \int (\sec ^2 x - 1) dx \\
        &= \tan x - x + C
    \end{aligned}$$

    \subsection*{32. $\int \tan^2 x \sec x dx$}

    We first solve $\int \sec^3 x dx$:
    $$\begin{aligned}
        \int \sec^3 x dx &= \int \sec x (\sec^2 x dx) \\
        &= \int \sec x d\tan x \\
        &= \sec x \tan x - \int \tan x d\sec x  \\
        &= \sec x \tan x - \int \tan^2 x \sec x dx \\
        &= \sec x \tan x - \int (\sec^2 x - 1) \sec x dx \\
        &= \sec x \tan x - \int \sec^3 x dx + \int \sec x dx \\
        &= \sec x \tan x - \int \sec^3 x dx + \ln |\sec x + \tan x|
    \end{aligned}$$

    $$\therefore \int \sec^3 x dx = \frac 1 2 (\sec x \tan x + \ln |\sec x + \tan x|) + C$$

    $$\begin{aligned}
        \int \tan^2 x \sec x dx &= \int \sec ^3 x - \sec x dx \\
        &= \frac 1 2(\sec x \tan x - \ln |\sec x + \tan x|) + C
    \end{aligned}$$

    \subsection*{53. $\int \sin 3x \sin 6x dx$}

    $$\begin{aligned}
        \int \sin 3x \sin 6x dx &= \int \sin 3x(2\sin 3x \cos 3x) dx\\
        &= \int 2\sin^2 3x \cos 3x dx\\
        &= \int 2(\cos 3x - \cos ^3 3x) dx \\
        &= \frac 2 3 \sin 3x - \int 2\cos^3 3x \\
        &= \frac 2 3 \sin 3x - 2\int (\cos 3x - \sin^2 3x \cos 3x) dx \\
        &= \frac 2 3 \sin 3x - \frac 2 3 \sin 3x + \frac 2 9 \sin^3 3x + C \\
        &= \frac 2 9 \sin ^3 3x + C
    \end{aligned}$$

    \section*{Exercise 7.3}

    \subsection*{6. $\int_0^3\frac{x}{\sqrt{36 - x^2}} dx$}

    $$\int_0^3\frac{x}{\sqrt{36 - x^2}}dx = \frac 1 2 \int_0^3 \frac{1}{\sqrt{36 - x^2}}dx^2$$

    let $x = 6u$, then

    $$\begin{aligned}
        \frac 1 2 \int_0^3 \frac{1}{\sqrt{36 - x^2}}dx^2 &= \frac 1 2 \int_0^{\frac 1 2}\frac{1}{\sqrt{36 - 36u^2}} d 36u^2 \\
        &= \int_0^{\frac 1 2 }\frac{72}{2 \times 6}\frac{1}{\sqrt{1 - u^2}} du \\
        &= 6\arcsin u \biggl|_0^{\frac 1 2} \\
        &= 6(\frac \pi 6 - 0) \\
        &= \pi
    \end{aligned}$$

    \subsection*{7. $\int_0^a \frac{dx}{(a^2 + x^2)^{\frac 3 2}}, \quad a > 0$}

    Let $x = a\tan t$, and $t = \arctan \frac x a$, then

    $$\begin{aligned}
        \int _0^a \frac{1}{(a^2 + x^2)^{\frac 3 2}}dx &= \int_0^{\frac \pi 4} \frac{1}{(a^2\sec^2 t)^{\frac 3 2}}a\sec^2 t dt \\
        &= \int_0^{\frac \pi 4} \frac{1}{a^2\sec t}dt \\
        &= \int_0^{\frac \pi 4} \frac{\cos t}{a^2}dt \\
        &= \frac{\sin t}{a^2} \biggl |_0^{\frac \pi 4} \\
        &= \frac{\sqrt 2}{2a^2}
    \end{aligned}$$

    \subsection*{8. $\int \frac{dt}{t^2\sqrt{t^2 - 16}}$}

    Let $t = 4\sec \theta$, then

    $$\begin{aligned}
        \int \frac{dt}{t^2\sqrt{t^2 - 16}} &= \int \frac{1}{16 \sec^2 \theta 4 \tan \theta} 4\sec \theta \tan \theta d\theta \\
        &= \int \frac{1}{16 \sec \theta} d\theta \\
        &= \int \frac{\cos \theta}{16} d\theta \\
        &= \frac{\sin \theta}{16} + C \\
        &= \frac{1}{16} \sqrt{t^2 - 16}{t^2} + C
    \end{aligned}$$

    \subsection*{13. $\int \frac{\sqrt{x^2 - 9}}{x^3} dx $}

    Let $x = 3\sec t$, then 
    $$\begin{aligned}
        \int \frac{\sqrt{x^2 - 9}}{x^3} dx &= \int \frac{3\tan t}{27\sec^3 t} 3 \sec t \tan t dt \\
        &= \int \frac{\tan ^2 t}{3\sec^2 t} dt \\
        &= \int \frac{\sin^2 t}{3} dt \\
        &= \int \frac{1 - \cos 2t}{6} dt \\
        &= \frac 1 6 t - \frac{\sin 2t}{3} + C \\
        &=  \frac 1 6 \arccos \frac 3 x - \frac{2}{x} \sqrt{1 - \frac{9}{x^2}} + C
    \end{aligned}$$

    \subsection*{23. $\int \sqrt{5 + 4x - x^2} dx $}

    $$\int \sqrt{5 + 4x - x^2} dx = \int \sqrt{9 -(x -2)^2} d(x - 2)$$

    Let $y = x - 2$, $y = 3 \sin t$, $t \in (0, \pi)$, then 

    $$\begin{aligned}
        \int \sqrt{5 + 4x - x^2} dx &= \int \sqrt{9 - y^2} dy \\
        &= 3\int \cos t 3 \cos t dt \\
        &= \int 9\frac{1 + \cos 2t}{2} dt \\
        &= \frac 9 2t + \frac 9 4 \sin 2t + C \\
        &= \frac 9 2 \arcsin \frac{x - 2}{3} + \frac 3 2(x - 2)\sqrt{1 - \frac{(x - 2)^2}{9}} + C
    \end{aligned}$$

    \subsection*{25. $\int \frac{x}{\sqrt{x^2 + x + 1}}dx $}

    $$\begin{aligned}
        \int \frac{x}{\sqrt{x^2 + x + 1}}dx &= \int \frac{x}{\sqrt{(x + \frac 1 2)^2 + \frac 3 4}} dx \\
    \end{aligned}$$

    Let $u = x + \frac 1 2$, then

    $$\begin{aligned}
        \int \frac{x}{\sqrt{x^2 + x + 1}}dx &= \int \frac{u - \frac 1 2}{\sqrt{u^2 + \frac 3 4}}du \\
    \end{aligned}$$

    Now substitue $u = \frac{\sqrt 3}{2} \tan \theta$, so

    $$\begin{aligned}
        \int \frac{x}{\sqrt{x^2 + x + 1}} dx &= \int \frac{\frac{\sqrt 3}{2}\tan \theta - \frac 1 2}{\frac{\sqrt 3}{2} \sec \theta}\frac{\sqrt 3}{2}\sec^2 \theta d\theta \\
        &= \int (\frac{\sqrt 3}{2} \tan\theta \sec \theta - \frac 1 2 \sec \theta) d\theta \\
        &= \frac{\sqrt{3}}{2}\sec \theta - \frac 1 2 \ln|\sec \theta + \tan \theta| + C
    \end{aligned}$$
\end{document}