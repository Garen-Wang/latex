\documentclass{article}
\usepackage{amsmath}
\usepackage{amsthm}
\usepackage{amssymb}
\usepackage{enumerate}
\usepackage{graphicx}

\begin{document}
  \title{Exercise 15.1}
  \author{Wang Yue from CS Elite Class}
  \date{\today}
  \maketitle


  \subsection*{14. }

  $$\begin{aligned}
    \iint_R \sqrt{9-y^2}dA &= \int_0^4 \int_0^2 \sqrt{9-y^2} dy dx \\
    &= \int_0^2 \int_0^4 \sqrt{9-y^2} dx dy \\
    &= 4\int_0^2 \sqrt{9-y^2} dy \\
  \end{aligned}$$

  Let $y = 3 \sin t$, where $t$, then

  $$\begin{aligned}
    4\int_0^2 \sqrt{9-y^2} dy &= 4\int_0^{\arcsin \frac 2 3} 3 |\cos t| 3\cos t dt \\
    &= 36\int_0^{\arcsin \frac 2 3} \cos^2 t dt \\
    &= 18\int_0^{\arcsin \frac 2 3} (1 + \cos 2t)dt \\
    &= 18(t + \frac 1 2 \sin 2t)\biggl|_0^{\arcsin \frac 2 3} \\
    &= 18(\arcsin \frac 2 3 + \frac 2 3 \cos \arcsin \frac 2 3) \\
    &= 18(\frac{41.810\pi}{180} + 0.4969) \\
    &= 22.079
  \end{aligned}$$

  \subsection*{17. }

  \begin{proof}
    $$\begin{aligned}
      \iint_R f(x, y)dA &= \iint_R k dA \\
      &= \int_a^b \int_c^d k dy dx \\
      &= \int_a^b k(d - c) dx \\
      &= k(b-a)(d-c)
    \end{aligned}$$
  \end{proof}

  \subsection*{18. }
  \begin{proof}
    $\because x \in [0, \frac 1 4]$, $\therefore \sin \pi x \in [0, \frac{\sqrt 2}{2}]$

    $\because y \in [\frac 1 4, \frac 1 2]$, $\therefore \cos \pi x \in [0, \frac{\sqrt 2}{2}]$

    $$\therefore 0 \leq \sin \pi x \cos \pi y \leq \frac{\sqrt 2}{2} \times \frac{\sqrt 2}{2} = \frac 1 2$$

    $\therefore $ by the conclusion of Exercise 17,

    $$0 \times \frac 1 4 \times \frac 1 4 \leq \int_0^{\frac 1 4}\int_{\frac 1 4}^{\frac 1 2}\sin \pi x \cos \pi y dy dx \leq \frac 1 2 \times \frac 1 4 \times \frac 1 4$$

    which is equivalent to

    $$0 \leq \iint_R \sin \pi x \cos \pi y dA \leq \frac{1}{32}$$
  \end{proof}

\end{document}
