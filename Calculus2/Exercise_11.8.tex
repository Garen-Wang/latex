\documentclass{article}
\usepackage{amsmath}
\usepackage{amsthm}
\usepackage{amssymb}
\usepackage{enumerate}
\usepackage{graphicx}

\begin{document}
    \title{Exercise 11.8}
    \author{Wang Yue from CS Elite Class}
    \date{\today}

    \maketitle

    \subsection*{3. $\sum_{n=1}^\infty (-1)^n nx^n$}

    Let $c_n = (-1)^nn \not = 0$, and $$\rho = \lim_{n\to\infty}|\frac{c_{n+1}}{c_n}| = \lim_{n\to\infty}\frac{n+1}{n} = 1$$

    Therefore, the radius of $\sum_{n=1}^\infty (-1)^n nx^n$ is $R = \frac{1}{\rho} = 1$

    When $x=-1$, $$\sum_{n=1}^\infty (-1)^n nx^n = \sum_{n=1}^\infty (-1)^{n} n (-1)^{n} = \sum_{n=1}^\infty n$$ which is divergent.

    When $x=1$, the series $$\sum_{n=1}^\infty (-1)^n nx^n = \sum_{n=1}^\infty (-1)^n n$$ which is also divergent.

    $\therefore $ the interval of convergence of the series is $(-1, 1)$.

    \subsection*{8. $\sum_{n=1}^\infty n^n x^n$}

    Let $c_n = n^n$, then $$\rho = \lim_{n\to\infty} \sqrt[n]{c_n} = \lim_{n\to\infty} n = \infty$$

    Therefore, the radius of $\sum_{n=1}^\infty n^n x^n$ is $R = 0$.

    Also, the interval of convergence of the series is $\{0\}$.

    \subsection*{9. $\sum_{n=1}^\infty (-1)^n \frac{n^2 x^n}{2^n}$}

    Let $c_n = (-1)^n \frac{n^2}{2^n}$, then $$\rho = \lim_{n\to\infty}| \frac{c_{n+1}}{c_n} | = \lim_{n\to\infty}\frac{(n+1)^2}{2n^2} = \lim_{n\to\infty}\frac{(n+1)}{2n} = \frac 1 2$$

    Therefore, the radius of $\sum_{n=1}^\infty (-1)^n \frac{n^2 x^n}{2^n}$ is $R = \frac{1}{\rho} = 2$

    When $x = 2$, $$\sum_{n=1}^\infty (-1)^n \frac{n^2 x^n}{2^n} = \sum_{n=1}^\infty (-1)^n n^2$$

    $\because \sum_{n=1}^\infty n^2$ is not decreasing and $\lim_{n\to\infty} n^2 = \infty$

    $\therefore \sum_{n=1}^\infty (-1)^n n^2$ is obviously divergent

    When $x = -2$, $$\sum_{n=1}^\infty (-1)^n \frac{n^2 x^n}{2^n} = \sum_{n=1}^\infty n^2$$ which is obviously divergent

    $\therefore$ the interval of convergence of the series is $(-2, 2)$.

    \subsection*{13. $\sum_{n=2}^\infty (-1)^n \frac{x^n}{4^n \ln n}$}

    Let $c_n = (-1)^n \frac{1}{4^n \ln n}$, then $$\lim_{n\to\infty}| \frac{c_{n+1}}{c_n} | = \lim_{n\to\infty} \frac{\ln n}{4\ln(n+1)} = \frac 1 4$$

    Therefore, the radius of $\sum_{n=2}^\infty (-1)^n \frac{x^n}{4^n \ln n}$ is $R = \frac{1}{\rho} = 4$

    When $x = 4$, $$\sum_{n=2}^\infty (-1)^n \frac{x^n}{4^n \ln n} = \sum_{n=2}^\infty (-1)^n \frac{1}{\ln n}$$

    Denote $\frac{1}{\ln n}$ to be $b_n$.

    $\because b_n$ is decreasing, $b_{n+1} < b_n$, and $$\lim_{n\to\infty}b_n = \lim_{n\to\infty} \frac{1}{\ln n} = 0$$

    $\therefore \sum_{n=2}^\infty (-1)^n \frac{1}{\ln n}$ is convergent

    When $x=-4$, $$\sum_{n=2}^\infty (-1)^n \frac{x^n}{4^n \ln n} = \sum_{n=2}^\infty \frac{1}{\ln n}$$

    $\because \ln n < n \quad \therefore \frac{1}{\ln n} > \frac{1}{n}$

    $$\therefore \sum_{n=2}^\infty \frac{1}{\ln n} > \sum_{n=2}^\infty \frac{1}{n}$$

    $\because \sum_{n=2}^\infty \frac 1 n$ is divergent

    $\therefore \sum_{n=2}^\infty \frac{1}{\ln n}$ is also divergent

    $\therefore$ the interval of convergence of the series is $(-4, 4]$

    \subsection*{20. $\sum_{n=1}^\infty \frac{(2x-1)^n}{5^n \sqrt n}$}

    First we convert the series to standard form:

    $$\sum_{n=1}^\infty \frac{(2x-1)^n}{5^n \sqrt n} = \sum_{n=1}^\infty \frac{(x - \frac 1 2)^n}{10^n \sqrt n}$$

    Let $c_n = \frac{1}{10^n \sqrt n}$, then $$\rho = \lim_{n\to\infty}| \frac{c_{n+1}}{c_n} | = \lim_{n\to\infty} \frac{\sqrt{n}}{10\sqrt{n+1}} = \frac{1}{10}$$

    Therefore, the radius of $\sum_{n=1}^\infty \frac{(x - \frac 1 2)^n}{10^n \sqrt n}$ is $R = \frac{1}{\rho} = 10$

    When $x = \frac 1 2 + 10 = \frac{21}{2}$, $$\sum_{n=1}^\infty \frac{(x-\frac 1 2)^n}{10^n \sqrt n} = \sum_{n=1}^\infty \frac{1}{\sqrt n}$$

    $\because \sum_{n=1}^\infty \frac{1}{\sqrt n}$ is a $p$-series, whose $p = \frac 1 2 < 1$
    
    $\therefore \sum_{n=1}^\infty \frac{1}{\sqrt n}$ is divergent

    When $x = \frac 1 2 - 10 = -\frac{19}{2}$, $$\sum_{n=1}^\infty \frac{(x - \frac 1 2)^n}{10^n \sqrt n} = \sum_{n=1}^\infty (-1)^n \frac{1}{\sqrt n}$$

    $\because \frac{1}{\sqrt{n+1}} < \frac{1}{\sqrt n}$ and $\lim_{n\to\infty}\frac{1}{\sqrt n} = 0$

    $\therefore \sum_{n=1}^\infty (-1)^n \frac{1}{\sqrt n}$ is convergent

    $\therefore$ the interval of convergence of the series is $[-\frac{19}{2}, \frac{21}{2})$

    \subsection*{30.}

    $\because \sum_{n=0}^\infty c_nx^n$ converges when $x = -4$ and diverges when $x = 6$

    $\therefore R \in [4, 6)$, where $R$ is the radius of $\sum_{n=0}^\infty c_n x^n$ 

    $$\therefore \lim_{n\to\infty} | \frac{c_{n+1}}{c_n} | = \frac{1}{R} \in [\frac 1 6, \frac 1 4)$$

    \begin{enumerate}[(a)]
        \item convergent
        
        $\because \lim_{n\to\infty}|\frac{c_{n+1}}{c_n}| < \frac 1 4 < 1$

        $\therefore \sum_{n=0}^\infty c_n$ is absolutely convergent, and therefore convergent

        \item divergent

        $$\because \lim_{n\to\infty}|\frac{8^{n+1}c_{n+1}}{8^nc_n}| = \lim_{n\to\infty}|\frac{8c_{n+1}}{c_n}| \geq 8 \times \frac 1 6 > 1$$

        $\therefore \sum_{n=0}^\infty c_n8^n$ is divergent

        \item convergent

        $$\because \lim_{n\to\infty} | \frac{c_{n+1}(-3)^{n+1}}{c_n(-3)^n} | = \lim_{n\to\infty} 3 | \frac{c_{n+1}}{c_n} | < 3 \times \frac 1 4 < 1$$

        $\therefore \sum_{n=0}^\infty c_n(-3)^n$ is convergent

        \item divergent

        $$\because \lim_{n\to\infty}| \frac{(-1)^{n+1} c_{n+1} 9^{n+1}}{(-1)^{n} c_{n} 9^{n}} | = \lim_{n\to\infty}9 | \frac{c_{n+1}}{c_n} | \geq 9 \times \frac 1 6 > 1$$

        $\therefore \sum_{n=0}^\infty (-1)^n c_n 9^n$ is divergent
    \end{enumerate}

\end{document}