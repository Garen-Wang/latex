\documentclass{article}
\usepackage{amsmath}
\usepackage{amsthm}
\usepackage{amssymb}
\usepackage{enumerate}
\usepackage{graphicx}

\begin{document}
    \title{Exercise 12.5}
    \author{Wang Yue from CS Elite Class}
    \date{March 30, 2021}
    \maketitle

    \subsection*{18. }

    Let $A(10, 3, 1), B(5, 6, -3)$, then $\overrightarrow{AB} = (-5, 3, -4)$

    Suppose $P$ is an arbitrary point in the line segment $AB$, then $$\overrightarrow{OP} = \overrightarrow{OA} + t\overrightarrow{AB} = (10-5t, 3+3t, 1-4t)$$

    where $0 \leq t \leq 1$.

    $\therefore$ parametric equations for the line segment is $$ x = 10 - 5t \quad y = 3 + 3t \quad z = 1 - 4t \quad 0 \leq t \leq 1$$

    \subsection*{26. }

    $\because$ the plane is perpendicular to the line $x=3t,y=2-t,z=3+4t$

    $\therefore \overrightarrow{n} = (3, -1, 4)$

    $\because a=3, b=-1,c=4,x_0=2,y_0=0,z_0=1$

    $\therefore$ an euqation of the plane is $3(x-2) -(y-0) + 4(z-1) = 0$, or
    
    $$3x-y+4z=10$$

    \subsection*{27. }

    $\because $the normal vector of the plane $5x-y-z=6$ is $\overrightarrow{n} = (5, -1, -1)$.

    $\therefore a=5, b=-1,c=-1, x_0=1, y_0=-1, z_0=-1$

    $\therefore$ an equation of the plane is $5(x-1) -(y+1) - (z+1) = 0$, or

    $$5x-y-z=7$$

    \subsection*{34.}

    Let $P(3t, 1 + t, 2 - t)$, then we know $P$ can be an arbitrary point in the line $x=3t, y=1+t, z=2-t$.
    
    Let $P_0 (1, 2, 3)$, then $\overrightarrow{P_0P} = (3t - 1, t - 1, - t - 1)$
    
    Suppose the normal vector $\overrightarrow{n} = (a, b, c)$, then

    $$\overrightarrow{n} \cdot \overrightarrow{P_0P} = (3t-1)a + (t-1)b - (t+1)c = 0$$

    When $t = 1$, we can get $2a - 2c = 0 \Rightarrow a=c$

    When $t=-1$, we can get $-4a - 2b=0 \Rightarrow b = -2a$

    Suppose $a=1$, then we know one of the normal vector is $\overrightarrow{n} = (1, -2, 1)$

    $\therefore a=1, b=-2, c=1, x_0 = 1, y_0 = 2, z_0 =3$

    $\therefore$ an equation of the plane is $(x - 1) - 2(y - 2) + (z - 3) = 0$, or 

    $$x-2y+z=0$$

    \subsection*{40.}

    The plane that passes through the line of intersection of the plane$x - z= 1$ and $y+2z=3$ can be represented as

    $$(x-z-1) + \lambda (y+2z-3) = 0$$

    which is equivalent to

    $$x + \lambda y + (2\lambda - 1)z -3\lambda - 1 = 0$$

    $\therefore$ the normal vector of the plane is $\overrightarrow{m}=(1, \lambda, 2\lambda - 1)$

    $\because$ the normal vector of $x+y-2z=1$ is $\overrightarrow{n} = (1, 1, -2)$

    $\because$ these two planes are perpendicular to each other

    $\therefore \overrightarrow{m}$ is perpendicular to $\overrightarrow{n}$

    $\therefore \overrightarrow{m} \cdot \overrightarrow{n} = 1 + \lambda - 4\lambda + 2 = 0$

    $\therefore$ we can solve that $\lambda = 1$

    $\therefore$ an equation of the plane is $x - z - 1 + y + 2z - 3 = 0$, or 

    $$x + y + z = 4$$

    \subsection*{49.}

    $$\because \left\{ \begin{array}{ll} x + y + z = 1 \\ x + z = 0 \end{array}\right. \Rightarrow \left\{ \begin{array}{ll} x+z=0 \\ y=-1 \end{array}\right.$$

    which can be rewritten as

    $$x = t \qquad y = -1 \qquad z = -t$$

    Let $\overrightarrow{v} = (1, 0, -1)$, then $\overrightarrow{v}$ is the direction vector of the line.

    $\therefore$ the direction numbers are 1, 0 and -1.

    \subsection*{50.}

    The normal vector of $x+y+z=0$ is $\overrightarrow{n} = (1, 1, 1)$

    Similarly, the normal vector of $x+2y+3z=1$ is $\overrightarrow{m} = (1, 2, 3)$

    $\therefore $ if $\theta$ is the angle between the planes, then
    
    $$\cos \theta = \frac{\overrightarrow{n} \cdot \overrightarrow{m}}{|\overrightarrow{n}| |\overrightarrow{m}|} = \frac{1 + 2 + 3}{\sqrt{3}\sqrt{14}} = \frac{2\sqrt 3}{\sqrt{14}}$$

    $$\therefore \theta = \arccos(\frac{2\sqrt 3}{\sqrt{14}}) \approx 22.2^\circ$$

    \subsection*{61.}

    Let $A(1, 0, -2), B(3, 4, 0)$, then we can get their midpoint $M(2, 2, -1)$.

    Plus, we know $\overrightarrow{AB} = (2, 4, 2)$.

    The plane in which every point is equidistant from $A$ and $B$ has normal vector $\overrightarrow{AB}$ and passes the point $M$. That is $2(x-2) + 4(y-2) + 2(z+1) = 0$. That is

    $$x + 2y + z = 5$$

    \subsection*{75.}

    \begin{proof}

        Let $A(x_1, y_1, z_1)$ be an arbitrary point in the plane $ax+by+cz+d_1=0$

        Then we can compute the distance between A and $ax+by+cz+d_2=0$ is 

        $$D = \frac{|ax_1 + by_1 + cz_1 + d_2|}{\sqrt{a^2 + b^2 + c^2}} = \frac{|-d_1 + d_2|}{\sqrt{a^2 + b^2 + c^2}} = \frac{|d_2 - d_1|}{\sqrt{a^2 + b^2 + c^2}}$$
        
    \end{proof}

    \subsection*{79.}

    Let $A(2, 0, -1), B(1, -1, 1), C(4, 1, 3)$

    Then $L_1$ can be represented as $$t\overrightarrow{OA} = (2t, 0, -t)$$

    Then $L_2$ can be represented as $$\overrightarrow{OB} + t\overrightarrow{BC} = (1, -1, 1) + (3t, 2t, 2t) = (3t + 1, 2t - 1, 2t + 1)$$

    Let $\overrightarrow{v_1} = \overrightarrow{OA} = (2, 0, -1), \overrightarrow{v_2} = \overrightarrow{BC} = (3, 2, 2)$

    Then $\overrightarrow{v_1} \times \overrightarrow{v_2} = \begin{vmatrix}
        i & j & k \\
        2 & 0 & -1 \\
        3 & 2 & 2 \\
    \end{vmatrix} = 2\overrightarrow{i} - 7\overrightarrow{j} + 4\overrightarrow{k}$

    Since $L_1$ and $L_2$ are skew, they can be viewed as lying on two parallel planes $P_1$ and $P_2$.

    An equation of $P_1$ is $2x -7y + 4k = 0$

    An equation of $P_2$ is $2(x - 1) - 7(y + 1) + 4(z - 1) = 0$ or $2x - 7y + 4z = 13$

    Therefore, the distance of $L_1$ and $L_2$ is equal to the distance from $O(0, 0, 0)$ to $2x - 7y + 4z = 13$. That is

    $$D = \frac{|0 - 0 + 0 - 13|}{\sqrt{2^2 + (-7)^2 + 4^2}} = \frac{13}{\sqrt{69}} \approx 1.565$$

    \subsection*{81.}

    \begin{proof}
        $\because ax + by + cz + d = 0$ can be rewritten as

        $$a(x+\frac d a) + b(y - 0) + c(z - 0) = 0$$

        which can be expressed as $(a, b, c) \cdot (x + \frac d a, y, z) = 0$

        Let $\overrightarrow{n} = (a, b, c), \overrightarrow{r} = (x, y, z), \overrightarrow{r_0} = (-\frac d a, 0, 0)$, then $(a, b, c) \cdot (x + \frac d a, y, z) = 0$ can be rewritten as
        
        $$\overrightarrow{n} \cdot (\overrightarrow{r} - \overrightarrow{r_0}) = 0$$

        which is called the vector equation of the plane.

        $\therefore ax+by+cz+d=0$ represents a plane and $(a, b, c)$ is normal vector.

    \end{proof}


\end{document}