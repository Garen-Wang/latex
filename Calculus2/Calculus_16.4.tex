\documentclass{article}
\usepackage{amsmath}
\usepackage{amsthm}
\usepackage{amssymb}
\usepackage{enumerate}
\usepackage{tikz}
\usepackage{graphicx}

\begin{document}
  \title{Exercise 16.4}
  \author{Wang Yue from CS Elite Class}
  \maketitle

  \subsection*{1. }

  $P(x, y) = x - y, Q(x, y) = x + y$

  Method 1: $\left\{ \begin{array}{ll} x = 2\cos t \\ y = 2 \sin t \end{array}\right.$

  $$\begin{aligned}
    \oint_C (x-y) dx + (x+y)dy &= \int_0^{2\pi} 4(\cos t - \sin t)(-\sin t) + 4(\cos t + \sin t)(\cos t) dt \\
    &= 4\int_0^{2\pi} \cos^2 t + \sin t \cos t - \sin t \cos t + \sin^2 t dt \\
    &= 8\pi
  \end{aligned}$$

  Method2: $$\frac{\partial Q}{\partial x} = 1, \frac{\partial P}{\partial y} = -1$$

  $$\oint_C (x-y)dx + (x+y)dy = \iint_D [1-(-1)] dA$$

  where $D = \{ (x, y) | x^2 + y^2 \leq 4 \}$

  $\therefore \iint_D 2dA = 2 \times 4\pi = 8\pi$

  \subsection*{3. }

  $P(x, y) = xy, Q(x, y) = x^2y^3$

  Method 1: $\left\{ \begin{array}{ll}  \end{array}\right.$

  Let $C_1, C_2, C_3$ be the path from $(0, 0)$ to $(1, 0)$, from $(1, 0)$ to $(1, 2)$ and from $(1, 2)$ to $(0, 0)$, respectively. Then

  $$\begin{aligned}
    \int_{C_1} xydx + x^2y^3 dy &= 0 \\
    \int_{C_2} xydx + x^2y^3 dy &= \int_0^2 t^3 dt = 4 \\
    \int_{C_3} xydx + x^2y^3 dy &= \int_1^0 2t^2 + 16t^5 dt = -\frac{10}{3} \\
    \oint_C xydx + x^2y^3 dy &= 0 + 4 - \frac{10}{3} = \frac{2}{3}
  \end{aligned}$$

  Method 2: $$\frac{\partial Q}{\partial x} = 2xy^3, \frac{\partial P}{\partial y} =  x$$

  $$\begin{aligned}
    \oint_{C} xydx + x^2y^3 dy &= \int_0^1 \int_0^{2x} (2xy^3 - x) dy dx \\
    &= \int_0^1 x(\frac{y^4}{2}-y)\biggl|_0^{2x} dx \\
    &= \int_0^1 (8x^5 - 2x^2) dx \\
    &=  (\frac{4x^6}{3} - \frac{2x^3}{3})\biggl|_0^1 \\
    &= \frac 2 3
  \end{aligned}$$

  \subsection*{7. }

  $P(x, y) = y + e^{\sqrt x} , Q(x, y) = 2x + \cos y^2$

  $$\frac{\partial Q}{\partial x} = 2, \frac{\partial P}{\partial y} = 1$$

  $$\begin{aligned}
    \therefore \int_C (y+e^{\sqrt x}) dx + (2x + \cos y^2) dy &= \int_0^1 \int_{x^2}^{\sqrt x} (2 - 1) dy dx \\
    &= \int_0^1 (x^{\frac 1 2} - x^2) dx \\
    &= (\frac{2x^{\frac 3 2}}{3} - \frac{x^3}{3})\biggl|_0^1 \\
    &= \frac 1 3
  \end{aligned}$$

  \subsection*{8. }

  $P(x, y) = y^4, Q(x, y) = 2xy^3$

  $$\frac{\partial Q}{\partial x} = 2y^3, \frac{\partial P}{\partial y} = 4y^3$$

  $$\therefore \int_C y^4 dx + 2xy^3 dy = \int_D (2y^3 - 4y^3) dA$$

  where $D = \{ (x, y) | \frac{x^2}{2} + y^2 \leq 1 \}$

  Let $\left \{ \begin{array}{ll} x = \sqrt{2} \cos t \\ y = \sin t \end{array}\right.$, then

  $$\begin{aligned}
    \int_D (-2y^3) dA &= -2\int_0^{2\pi} \int_0^1 \sin^3 t \sqrt 2 r dr dt \\
    &= -2\sqrt 2 \int_0^{2\pi} \sin^3 t (\frac{r^2}{2})\biggl|_0^1 dt \\
    &= \sqrt 2 \int_0^{2\pi} (1-\cos^2 t) d\cos t \\
    &= \sqrt 2 (\cos t - \frac{\cos^3 t}{3})\biggl|_0^{2\pi} \\
    &= 0
  \end{aligned}$$

  \subsection*{12. }

  $P(x, y) = e^{-x} + y^2, Q(x, y) = e^{-y} + x^2$

  $$\frac{\partial Q}{\partial x} = 2x, \frac{\partial P}{\partial y} = 2y$$

  % Let $C_1$ be the arc of the curve $y = \cos x$ from $(-\frac \pi 2, 0)$ to $(\frac \pi 2, 0)$.

  % Let $C_2$ be the line segment from $(\frac \pi 2, 0)$ to $(-\frac \pi 2, 0)$. Then
  $$\begin{aligned}
    \oint_{C} \overrightarrow{F} \cdot d\overrightarrow{r} &= -\iint_D (2x-2y) dA
  \end{aligned}$$

  where $D = \{ (x, y) | -\frac \pi 2 \leq x \leq \frac \pi 2, 0 \leq y \leq \cos x \}$

  $$\begin{aligned}
    \oint_C \overrightarrow{F} \cdot d\overrightarrow{r} &= 2\int_{-\frac \pi 2}^{\frac \pi 2} \int_0^{\cos x} (y - x) dy dx \\
    &= 2\int_{-\frac \pi 2}^{\frac \pi 2} (\frac{y^2}{2} - xy)\biggl|_0^{\cos x} dx \\
    &= 2\int_{-\frac \pi 2}^{\frac \pi 2} (\frac{\cos^2 x}{2} - x \cos x) dx \\
    &= 4\int_0^{\frac \pi 2} \frac{\cos^2 x}{2} dx \\
    &= 2\int_0^{\frac \pi 2} \frac{1 + \cos 2x}{2} dx \\
    &= \int_0^{\frac \pi 2} (x + \frac 1 2 \sin 2x)\biggl|_0^{\frac \pi 2} \\
    &= \frac \pi 2
  \end{aligned}$$

  \subsection*{13. }

  $P(x, y) = y - \cos y, Q(x, y) = x \sin y$

  $$\frac{\partial Q}{\partial x} = \sin y, \frac{\partial P}{\partial y} = 1 + \sin y$$

  Let $D = \{ (x, y) | (x-3)^2 + (y+4)^2 \leq 4 \}$, then

  $$\begin{aligned}
    \int_C \overrightarrow{F} \cdot d\overrightarrow{r} &= -\iint_D (\sin y - 1 - \sin y) dA \\
    &= \iint_D dA = 4\pi
  \end{aligned}$$
  
  \subsection*{19. }

  $$\begin{aligned}
    A = \oint_C x dy &= \int_{2\pi}^{0} (t-\sin t)\sin t dt \\
    &= -\int_{2\pi}^0 t d\cos t - \int_{2\pi}^0 \frac{1 - \cos 2t}{2} dt \\
    &= -(t\cos t)\biggl|_{2\pi}^0 + \int_{2\pi}^0 \cos t dt - (\frac{t}{2} - \frac{\sin 2t}{4})\biggl|_{2\pi}^0 \\
    &= 2\pi + 0 - (-\pi) = 3\pi
  \end{aligned}$$

  \subsection*{27. }

  Let $C'$ be a counterclockwise-oriented circle with center the origin and radius $a$, where

  $$P(x, y) = \frac{2xy}{(x^2+y^2)^2}, Q(x, y) = \frac{y^2-x^2}{(x^2+y^2)^2}$$

  $$\frac{\partial Q}{\partial x} = \frac{-2x(x^2+y^2)^2 - (y^2-x^2)2(x^2+y^2)2x}{(x^2+y^2)^4} = \frac{-2x(x^2+y^2) + 4x(x^2-y^2)}{(x^2+y^2)^3} = \frac{2x^3-6xy^2}{(x^2+y^2)^3}$$

  $$\frac{\partial P}{\partial y} = \frac{2x(x^2+y^2)^2 - 4xy(x^2+y^2) 2y}{(x^2+y^2)^4} = \frac{2x^3 - 6xy^2}{(x^2+y^2)^3}$$

  Let $D$ be the region bounded by $C$ and $C'$, then

  $$\begin{aligned}
    \int_C Pdx + Qdy + \int_{-C'} Pdx + Qdy &= \iint_D (\frac{\partial Q}{\partial x} - \frac{\partial P}{\partial y})dA = 0
  \end{aligned}$$

  $$\therefore \int_C Pdx + Qdy = \int_{C'} Pdx + Qdy$$

  Let a parametric equation of $C'$ be $\left\{ \begin{array}{ll} x = a \cos t \\ y = a \sin t \end{array}\right.$, then

  $$\begin{aligned}
    \therefore \int_C \overrightarrow{F} \cdot d\overrightarrow{r} &= \int_{C'}\overrightarrow{F} \cdot d\overrightarrow{r} \\
    &= \int_0^{2\pi} \overrightarrow{F}(\overrightarrow{r}(t)) \cdot \overrightarrow{r}'(t) dt \\
    &= \int_0^{2\pi} \frac{2a^2 \sin t \cos t}{a^4}(-a\sin t) + \frac{a^2(\sin^2 t - \cos^2 t)}{a^4}(a\cos t) dt \\
    &= \int_0^{2\pi} -\frac{2}{a} \sin^2 t \cos t + \frac{1}{a}(2\sin^2 t - 1) \cos t dt \\
    &= \int_0^{2\pi} -\frac 1 a \cos t dt \\
    &= -\frac 1 a (\sin t)\biggl|_0^{2\pi} \\
    &= 0
  \end{aligned}$$

  \subsection*{31. }

  \begin{proof}

    $\because \left\{ \begin{array}{ll} x = g(u, v) \\ y = h(u, v) \end{array}\right.$

    $$\begin{aligned}
      \therefore \iint_R dx dy &= \frac 1 2 \int_{\partial R} xdy - ydx \\
      &= \frac 1 2 \int_{\partial S} g(u, v) (\frac{\partial h}{\partial u} du + \frac{\partial h}{\partial v} dv) - h(u, v) (\frac{\partial g}{\partial u} du + \frac{\partial g}{\partial v} dv) \\
      &= \frac 1 2 \int_{\partial S} [g(u, v)\frac{\partial h}{\partial u} - h(u, v) \frac{\partial g}{\partial u}] du + [g(u, v) \frac{\partial h}{\partial v} - h(u, v) \frac{\partial g}{\partial v}] dv \\
    \end{aligned}$$

    Let $P(u, v) = g(u, v)\frac{\partial h}{\partial u} - h(u, v)\frac{\partial g}{\partial u}, Q(u, v) = g(u, v) \frac{\partial h}{\partial v} - h(u, v) \frac{\partial g}{\partial v}$, then

    $$\frac{\partial Q}{\partial u} = \frac{\partial g}{\partial u}\frac{\partial h}{\partial v} + g(u, v)\frac{\partial^2 h}{\partial u \partial v} - \frac{\partial h}{\partial u}\frac{\partial g}{\partial v} - h(u, v) \frac{\partial^2 g}{\partial u \partial v}$$

    $$\frac{\partial P}{\partial v} = \frac{\partial g}{\partial v}\frac{\partial h}{\partial u} + g(u, v)\frac{\partial^2 h}{\partial v \partial u} - \frac{\partial h}{\partial v}\frac{\partial g}{\partial u} - h(u, v) \frac{\partial^2 g}{\partial v \partial u}$$

    $$\begin{aligned}
      \frac 1 2 \int_{\partial S} P(u, v) du + Q(u, v) dv &= \frac 1 2 \int_S (\frac{\partial Q}{\partial u} - \frac{\partial P}{\partial v}) du dv \\
      &= \frac 1 2 \int_S (2\frac{\partial g}{\partial u} \frac{\partial h}{\partial v} - \frac{\partial g}{\partial v}\frac{\partial h}{\partial u}) du dv \\
      &= \int_S \frac{\partial g}{\partial u} \frac{\partial h}{\partial v} - \frac{\partial g}{\partial v}\frac{\partial h}{\partial u} du dv \\
      &= \int_S \biggl| \frac{\partial(g, h)}{\partial (u, v)} \biggl| du dv \\
      &= \int_S \biggl| \frac{\partial(x, y)}{\partial (u, v)} \biggl| du dv \\
    \end{aligned}$$
  
  \end{proof}

\end{document}