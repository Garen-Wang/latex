\documentclass{article}
\usepackage{amsmath}
\usepackage{amsthm}
\usepackage{amssymb}
\usepackage{enumerate}
\usepackage{graphicx}

\begin{document}
    \title{Exercise 11.6}
    \author{Wang Yue from CS Elite Class}
    \date{\today}

    \maketitle

    \subsection*{6. $\sum_{n=0}^\infty \frac{(-3)^n}{(2n+1)!}$}

    We check whether $\sum_{n=0}^\infty \frac{3^n}{(2n+1)!}$ is convergent or not.

    By the ratio test, we have:

    $$\frac{\frac{3^{n+1}}{(2n+3)!}}{\frac{3^n}{(2n+1)!}} = \frac{3}{(2n+2)(2n+3)} \leq \frac{3}{2 \times 3} = \frac 1 2 < 1$$

    $\therefore \sum_{n=0}^\infty \frac{3^n}{(2n+1)!}$ is convergent

    $\therefore \sum_{n=0}^\infty \frac{(-3)^n}{(2n+1)!}$ is absolutely convergent

    \subsection*{7. $\sum_{k=1}^\infty k(\frac 2 3)^k$}

    $\because$ By the limit ratio test, we have:

    $$\lim_{k\to\infty} \frac{(k+1)(\frac 2 3)^{k+1}}{k(\frac 2 3)^k} = \lim_{k \to \infty} \frac 2 3 \frac{k+1}{k} = \frac 2 3 < 1$$

    $\therefore \sum_{k=1}^\infty k(\frac 2 3)^k$ is convergent

    $\because$ each term of $\sum_{k=1}^\infty k(\frac 2 3)^k$ is positive term

    $\therefore \sum_{k=1}^\infty k(\frac 2 3)^k$ is absolutely convergent

    \subsection*{10. $\sum_{n=1}^\infty (-1)^n \frac{n}{\sqrt{n^3+2}}$}

    Let $a_n = (-1)^n \frac{n}{\sqrt{n^3 + 2}}, b_n = \frac{n}{\sqrt{n^3 + 2}} = \frac{1}{\sqrt{n + \frac{2}{n^2}}}, c_n = n + \frac{2}{n^2}$

    Let $f(x) = x + \frac{2}{x^2}, x \geq 1$

    $\because$ when $x < 2, f(1) = 1 + 2 = 3 > f(2) = 2 + \frac 1 2 = \frac 5 2$

    $\because$ when $x \geq 2, f'(x) = 1 - \frac{4}{x^3} > \frac{1}{2} > 0, f(x)$ is increasing

    $\therefore \forall n \in N_+, c_n$ is increasing

    $\therefore \forall n \in N_+, b_n$ is decreasing, $b_{n+1} < b_n$

    $\because \lim_{n\to\infty} c_n = \lim_{n\to\infty}\frac{n^3+2}{n^2} = \lim_{n\to\infty}\frac{3n^2}{2n} = \infty$

    $\therefore \lim_{n\to\infty} b_n = 0$

    $\therefore \sum_{n=1}^\infty a_n = \sum_{n=1}^\infty (-1)^n \frac{n}{\sqrt{n^3 + 2}}$ is absolutely convergent

    \subsection*{11. $\sum_{n=1}^\infty \frac{(-1)^n e^{\frac 1 n}}{n^3}$}

    Let $a_n = \frac{(-1)^n e^{\frac 1 n}}{n^3}, b_n = \frac{e^{\frac 1 n}}{n^3}$

    $$\lim_{n\to\infty}|\frac{a_{n+1}}{a_n}| = \lim_{n\to\infty}\frac{n^3 e^{\frac{1}{n+1}}}{(n+1)^3 e^{\frac{1}{n}}} = \lim_{n\to\infty}\frac{1}{(1+\frac 1 n)^3 e^{\frac{1}{n(n+1)}}} = 1$$

    $\therefore$ TODO is inconclusive.

    $\because$ when $n$ increases, $e^{\frac 1 n}$ is decreasing, $n^3$ is increasing

    $\therefore b_n$ is decreasing, $b_{n+1} < b_n$

    $$\because \lim_{n\to\infty}\frac{e^{\frac 1 n}}{n^3} = \frac{0}{\infty} = 0$$

    $\therefore \sum_{n=1}^\infty \frac{(-1)^n e^{\frac 1 n}}{n^3}$ is absolutely convergent


    \subsection*{13. $\sum_{n=1}^\infty \frac{10^n}{(n+1)4^{2n+1}}$}

    $a_n = \frac{10^n}{(n+1)4^{2n+1}}$

    $$\lim_{n\to\infty}|\frac{a_{n+1}}{a_n}| = \lim_{n\to\infty} \frac{10(n+1)}{16(n+2)} = \lim_{n\to\infty}\frac{10}{16} = \frac 5 8 < 1$$

    $\therefore \sum_{n=1}^\infty a_n$ is absolutely convergent

    \subsection*{17. $\sum_{n=2}^\infty \frac{(-1)^n}{\ln n}$}

    Let $a_n = \frac{(-1)^n}{\ln n}, b_n = \frac{1}{\ln n}, n \geq 2$

    Obviously, we can know $b_n$ is decrasing, $b_{n+1} < b_n$, when $n \geq 2$

    $$\because \lim_{n\to\infty}b_n = \lim_{n\to\infty}\frac{1}{\ln n} = 0$$

    $\therefore \sum_{n=2}^\infty a_n$ is convergent

    $\because \ln n < n$

    $$\therefore \sum_{n=2}^\infty \frac{1}{\ln n} > \sum_{n=2}^\infty \frac{1}{n}$$

    $\because \sum_{n=2}^\infty \frac 1 n$ is divergent

    $\therefore \sum_{n=2}^\infty b_n$ is also divergent

    $\therefore $ it is conditionally convergent.

    \subsection*{21. $\sum_{n=1}^\infty (\frac{n^2+1}{2n^2+1})^n$}

    Let $a_n = (\frac{n^2+1}{2n^2+1})^n$

    $$\lim_{n\to\infty}\sqrt[n]{a_n} = \lim_{n\to\infty}\frac{n^2+1}{2n^2+1} = \lim_{n\to\infty}\frac{2n}{4n} = \frac 1 2 < 1$$

    $\therefore \sum_{n=1}^\infty (\frac{n^2+1}{2n^2+1})^n$ is absolutely convergent and therefore convergent

    \subsection*{23. $\sum_{n=1}^\infty (1 + \frac{1}{n})^{n^2}$}

    Let $a_n = (1 + \frac{1}{n})^{n^2}$

    $$\lim_{n\to\infty}\sqrt[n]{a_n} = \lim_{n\to\infty}(1 + \frac 1 n)^n = e > 1$$

    $\therefore \sum_{n=1}^\infty (1+\frac 1 n)^{n^2}$ is divergent

    \subsection*{32. A series $\sum a_n$ is defined by the equations $$a_1 = 1 \qquad a_{n+1} = \frac{2 + \cos n}{\sqrt n}a_n$$Determine whether $\sum a_n$ converges or diverges.}

    $$\lim_{n\to\infty}|\frac{a_{n+1}}{a_n}| = \lim_{n\to\infty}\frac{2 + \cos n}{\sqrt n} = 0$$

    $\therefore \sum a_n$ is absolutely convergent

    $\therefore \sum a_n$ is convergent

    \subsection*{34. $\sum_{n=1}^\infty \frac{(-1)^n n!}{n^n b_1 b_2 \cdots b_n}$}

    Let $a_n = \frac{(-1)^n n!}{n^n b_1 b_2 \cdots b_n}$

    $$\lim_{n\to\infty}|\frac{a_{n+1}}{a_n}| = \lim_{n\to\infty}\frac{n^n}{(n+1)^{n} b_{n+1}} = \lim_{n\to\infty}\frac{1}{(1 + \frac 1 n)^n} \lim_{n\to\infty}\frac{1}{b_{n+1}} = \frac{1}{\frac{e}{2}} = \frac 2 e < 1$$

    $\therefore\sum_{n=1}^\infty \frac{(-1)^n n!}{n^n b_1 b_2 \cdots b_n}$ is absolutely convergent

    $\therefore\sum_{n=1}^\infty \frac{(-1)^n n!}{n^n b_1 b_2 \cdots b_n}$ is convergent

    \subsection*{38. }

    \begin{enumerate}[(a)]
        \item \begin{proof}
            
            $$\begin{aligned}
                \because\{r_n\}& \textrm{ is decreasing} \\
                \therefore R_n &= a_{n+1} + a_{n+2} + a_{n+3} + \cdots \\
                &\leq a_{n+1} + r_{n+1}a_{n+1} + r_{n+1}^2a_{n+1} + \cdots \\
                &= \sum_{k=1}^\infty a_{n+1}r_{n+1}^{k-1} \\
                &= \frac{a_{n+1}}{1-r_{n+1}}
            \end{aligned}$$

        \end{proof}
        \item \begin{proof}
            
            $$\begin{aligned}
                \because \{r_n\}& \textrm{ is increasing} \\
                \therefore R_n &= a_{n+1} + a_{n+2} + a_{n+3} + \cdots \\
                &\leq a_{n+1} + a_{n+1}\lim_{n\to\infty}r_n + a_{n+1}(\lim_{n\to\infty}r_n)^2 + \cdots \\
                &= \sum_{k=1}^\infty a_{n+1}L^{k-1} \\
                &= \frac{a_{n+1}}{1-L}
            \end{aligned}$$
        \end{proof}
    \end{enumerate}

\end{document}